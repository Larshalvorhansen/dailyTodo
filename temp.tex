\section{Question 5} \subsection{Part (a)} To determine the behavior of the
nominal exchange rate, we need to analyze both the real exchange rate and price
level dynamics. \textbf{Real Exchange Rate Dynamics:} The real exchange rate is
determined by: \[ q_{N/S} = \frac{1}{\alpha} \left( 1 - \beta \frac{Y_N}{Y_S}
\right) \] Taking the time derivative and applying the chain rule: \[
\frac{\dot{q}_{N/S}}{q_{N/S}} = -\frac{\beta \frac{Y_N}{Y_S}}{1 - \beta
\frac{Y_N}{Y_S}} \cdot \left( \frac{\dot{Y}_N}{Y_N} - \frac{\dot{Y}_S}{Y_S}
\right) \] Let $g_N = 0.032$ and $g_S = 0.024$. The growth rate of the real
exchange rate is: \[ \frac{\dot{q}_{N/S}}{q_{N/S}} = -\frac{\beta
\frac{Y_N}{Y_S}}{1 - \beta \frac{Y_N}{Y_S}} \cdot (0.032 - 0.024) = -\frac{\beta
\frac{Y_N}{Y_S}}{1 - \beta \frac{Y_N}{Y_S}} \cdot 0.008 \] Since $\beta > 0$ and
$\frac{Y_N}{Y_S} > 0$, and assuming $\beta \frac{Y_N}{Y_S} < 1$ (for stability),
we have: \[ \frac{\dot{q}_{N/S}}{q_{N/S}} < 0 \] This means the real exchange
rate is \textbf{decreasing} (Norwegian basket becomes cheaper relative to
Swedish basket). \textbf{Nominal Exchange Rate Dynamics:} The relationship
between nominal and real exchange rates is: \[ q_{N/S} = E_{N/S} \cdot
\frac{P_S}{P_N} \] Taking logarithmic derivatives: \[
\frac{\dot{q}_{N/S}}{q_{N/S}} = \frac{\dot{E}_{N/S}}{E_{N/S}} +
\frac{\dot{P}_S}{P_S} - \frac{\dot{P}_N}{P_N} \] Given $\frac{\dot{P}_N}{P_N} =
\frac{\dot{P}_S}{P_S} = 0.02$, we have: \[ \frac{\dot{E}_{N/S}}{E_{N/S}} =
\frac{\dot{q}_{N/S}}{q_{N/S}} + 0.02 - 0.02 = \frac{\dot{q}_{N/S}}{q_{N/S}} < 0
\] \textbf{Conclusion:} The Norwegian krone is expected to \textbf{appreciate}
in nominal terms. The rate of appreciation equals: \[
\left|\frac{\dot{E}_{N/S}}{E_{N/S}}\right| = \frac{\beta \frac{Y_N}{Y_S}}{1 -
\beta \frac{Y_N}{Y_S}} \cdot 0.008 \] The exact rate depends on the parameters
$\alpha$, $\beta$ and the output ratio, but it is driven by the differential in
real output growth (0.8\% per year). \subsection{Part (b)} Now we have different
inflation rates: $\frac{\dot{P}_N}{P_N} = 0.02$ in Norway and
$\frac{\dot{P}_S}{P_S} = 0.03$ in Sweden. \textbf{Real Exchange Rate:} The real
exchange rate dynamics remain unchanged since they depend only on real output
growth: \[ \frac{\dot{q}_{N/S}}{q_{N/S}} = -\frac{\beta \frac{Y_N}{Y_S}}{1 -
\beta \frac{Y_N}{Y_S}} \cdot 0.008 < 0 \] \textbf{Nominal Exchange Rate:} Using
the same relationship: \[ \frac{\dot{E}_{N/S}}{E_{N/S}} =
\frac{\dot{q}_{N/S}}{q_{N/S}} + \frac{\dot{P}_S}{P_S} - \frac{\dot{P}_N}{P_N} \]
\[ \frac{\dot{E}_{N/S}}{E_{N/S}} = \frac{\dot{q}_{N/S}}{q_{N/S}} + 0.03 - 0.02 =
\frac{\dot{q}_{N/S}}{q_{N/S}} + 0.01 \] \textbf{Conclusion:} The nominal
exchange rate behavior depends on the relative magnitudes. There are two
offsetting effects: \begin{itemize} \item Real depreciation of Norwegian basket:
$\frac{\dot{q}_{N/S}}{q_{N/S}} < 0$ (pushes toward NOK appreciation) \item
Higher Swedish inflation: $+0.01$ (pushes toward NOK appreciation by 1\% per
year) \end{itemize} Both effects work in the same direction. The Norwegian krone
\textbf{appreciates} at a faster rate than in part (a): \[
\left|\frac{\dot{E}_{N/S}}{E_{N/S}}\right| = \frac{\beta \frac{Y_N}{Y_S}}{1 -
\beta \frac{Y_N}{Y_S}} \cdot 0.008 + 0.01 \] The appreciation rate is 1
percentage point higher than in scenario (a). \subsection{Part (c)}
\textbf{Economic Intuition:} In both scenarios, Norway's faster real output
growth (3.2\% vs 2.4\%) creates a fundamental pressure for the Norwegian krone
to appreciate in real terms. This occurs because: \begin{itemize} \item Higher
Norwegian output growth means increased supply of Norwegian goods relative to
Swedish goods \item According to the demand-supply equilibrium, this requires a
lower relative price for Norwegian goods (real appreciation of NOK) to clear
markets \item The real exchange rate $q_{N/S}$ decreases, meaning Norwegian
baskets become relatively cheaper \end{itemize} \textbf{Scenario (a) --- Equal
Inflation (2\% in both countries):} When inflation rates are equal, the nominal
exchange rate movements perfectly reflect the real exchange rate changes. The
NOK appreciates nominally at the same rate as it appreciates in real terms. This
is pure purchasing power parity (PPP) at work: with no inflation differential,
nominal and real exchange rate movements coincide. \textbf{Scenario (b) ---
Higher Swedish Inflation (3\% vs 2\%):} The higher Swedish inflation adds an
additional channel for NOK appreciation. Now we have two reinforcing effects:
\begin{itemize} \item \textbf{Real effect:} Same as scenario (a) --- faster
Norwegian growth requires real NOK appreciation \item \textbf{Monetary effect:}
Higher Swedish inflation means Swedish krona loses purchasing power faster than
Norwegian krone domestically, requiring additional nominal NOK appreciation to
maintain purchasing power parity \end{itemize} The 1 percentage point inflation
differential directly translates to an additional 1\% annual nominal
appreciation of the NOK beyond the real appreciation effect. \subsection{Part
(d)} To keep the nominal exchange rate constant ($\frac{\dot{E}_{N/S}}{E_{N/S}}
= 0$), Swedish authorities need to offset the natural appreciation pressures.
\textbf{Scenario (a):} Required condition: \[ 0 = \frac{\dot{q}_{N/S}}{q_{N/S}}
+ \frac{\dot{P}_S}{P_S} - \frac{\dot{P}_N}{P_N} \] Since
$\frac{\dot{q}_{N/S}}{q_{N/S}} < 0$ and $\frac{\dot{P}_S}{P_S} =
\frac{\dot{P}_N}{P_N} = 0.02$, the authorities must intervene to change the real
exchange rate dynamics. Options include: \begin{itemize} \item \textbf{Monetary
intervention:} Not directly effective here since the model assumes money
neutrality in the long run and prices are already at their equilibrium inflation
rates \item \textbf{Modify inflation differential:} Swedish authorities could
implement more expansionary monetary policy to increase Swedish inflation above
2\%, creating a nominal depreciation force to offset the real appreciation. They
would need $\frac{\dot{P}_S}{P_S} = 0.02 - \frac{\dot{q}_{N/S}}{q_{N/S}}$ \item
\textbf{Supply-side policies:} Implement policies to boost Swedish output growth
or constrain Norwegian output growth to eliminate the growth differential
(though this is beyond Swedish control and economically costly) \end{itemize}
\textbf{Scenario (b):} Required condition is the same, but now: \[
\frac{\dot{q}_{N/S}}{q_{N/S}} + 0.03 - 0.02 = \frac{\dot{q}_{N/S}}{q_{N/S}} +
0.01 \] Since both the real effect and inflation differential push toward NOK
appreciation, Swedish authorities need even stronger interventions:
\begin{itemize} \item \textbf{Reduce Swedish inflation:} Implement
contractionary monetary policy to reduce Swedish inflation from 3\% to
approximately $2\% - \frac{\dot{q}_{N/S}}{q_{N/S}}$. This would eliminate the
inflation differential's contribution to NOK appreciation \item \textbf{However,
this creates a trade-off:} Lower inflation might slow Swedish output growth
further, potentially worsening the real exchange rate pressure \item
\textbf{Most feasible approach:} Coordinate with Norwegian authorities for
Norway to increase its inflation rate, or Sweden to decrease its inflation rate,
such that: \[ \frac{\dot{P}_S}{P_S} - \frac{\dot{P}_N}{P_N} =
-\frac{\dot{q}_{N/S}}{q_{N/S}} \] \end{itemize} \textbf{General Note:} In both
scenarios, maintaining a fixed nominal rate against fundamental economic forces
(output growth differentials) is costly and may not be sustainable in the long
run without addressing the underlying real economic differences.
